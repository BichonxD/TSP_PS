\documentclass{article}

% Mise en page
\usepackage[scale=0.75]{geometry}
% Pied de page
\usepackage{fancyhdr}
\pagestyle{fancy}
\renewcommand{\headrulewidth}{0pt}
% Interligne après les paragraphes
\setlength{\parskip}{1.5ex}

% Langues
\usepackage[french]{babel}
\usepackage[utf8]{inputenc}
\usepackage[T1]{fontenc}

% Images
\usepackage{graphicx}
\usepackage{rotating}

\begin{document}

\newcommand{\HRule}{\rule{\linewidth}{0.5mm}}

\begin{titlepage}
\begin{center}

\includegraphics[width=0.5\textwidth]{polytech}~\\[1cm]

\textsc{\LARGE Programmation Stochastique}\\[1.5cm]

\textsc{\Large Projet d'implémentation de l'algorithme du Recuit Simulé}\\[0.5cm]

% Title
\HRule \\[0.4cm]
{ \huge \bfseries Traveling Salesman Problem \\[0.4cm] }
\HRule \\[1.5cm]

% Author and supervisor
\begin{minipage}{0.4\textwidth}
\begin{flushleft} \large
\emph{Auteur:}\\
Jeremie \textsc{Bosom}\\
Erwan \textsc{Chaussy}
\end{flushleft}
\end{minipage}
\begin{minipage}{0.4\textwidth}
\begin{flushright} \large
\emph{Professeur:} \\
M.Abdel \textsc{Lisser}
\end{flushright}
\end{minipage}

\vfill

% Bottom of the page
{\large \today}

\end{center}
\end{titlepage}


\rhead{Programmation Stochastique - Document de synthèse}
\lfoot{\includegraphics[scale=0.3]{../polytech.jpg}}
\rfoot{BOSOM - CHAUSSY}

\section{Problématique}

Nous avons eu plusieurs problèmes pour améliorer nos résultats lors de ce projet.
C'est pour cela que nous avons implémentés des options afin de pouvoir tester plus facilement les différentes possibilités que nous avions.
\par
La première question que nous nous sommes posée fut de savoir s'il valait mieux utiliser l'algorithme du Plus Proche Voisin ou de la Plus Proche Insertion, afin d'obtenir une solution acceptable de base.
Après avoir répondu à cette question, nous nous sommes demandés quelle était l'influence de notre algorithme pour trouver une solution voisine sur les résultats du recuit simulé.
Enfin nous nous sommes interrogés sur l'influence des paramètres du recuit sur nos solutions.
 
Ce document a pour but de retracer notre réflexion sur comment améliorer notre algorithme et les résultats que nous avons obtenus.
 
\section{PPV ou PPI}
 
Nous avons implémentés les algorithmes du Plus Proche Voisin et de la Plus Proche Insertion que nous abrégerons respectivement PPV et PPI.
Nous pensions, de prime abord, que la PPI nous permettrait d'obtenir un cycle de base plus optimisé que le PPV.
Néanmoins, après quelques tests, il s'est avéré que ce n'est pas le cas.
Non seulement la PPI est plus gourmande au niveau du temps que le PPV, mais en plus la PPI obtient des résultats qui sont sensiblement moins bon que le PPV.
C'est pourquoi nous avons décidé d'utiliser le PPV.
\par
Afin d'améliorer le PPV, nous avons décidé d'essayer d'utiliser du multi-thread.
Ainsi nous avons implémenté le Plus Proche Voisin Threadé, abrégé PPVT, qui fait du multi-thread pour trouver le point le plus proche du point actuellement étudié.
Nous avons ainsi réussi à accélérer l'acquisition d'un cycle de base.
 
\section{Influence de l'algorithme pour trouver une solution voisine}
 
Nous avons choisi d'utiliser un algorithme proche de celui du 2-opt pour trouver une solution voisine pour la simple raison qu'avec les autres algorithmes que nous avons implémentés nous n'avions aucune amélioration.
Vu le peu d'intérêt de ces résultats nous ne les montrerons pas ici.
Néanmoins, nous avons laissé les algorithmes que nous avons implémentés dans notre code source.
Pour voir les résultats qu'ils permettent, il suffit de lancer le programme en mode debuggage avec l'option "-verbose" en ayant changé l'appel à ces fonctions.
 
\section{Influence des paramètres du recuit simulé}
 
Lors de nos premiers essais les paramètres n'étaient pas du tout optimaux.
Nous obtenions des résultats très proches de ceux donnés par le PPVT ou le PPV.
Avoir la possibilité de modifier ces réglages grâce aux options que nous avions écrites nous fut très pratique.

\clearpage
\section{Résultats finaux}
 
Voici les résultats que nous avons obtenus et les pourcentages d'erreur par rapport aux valeurs optimales.
\begin{center}
\begin{tabular}{|l|c|c|c|}
\hline
Instance & Coûts optimaux & Résultats & Différence (en \%)\\
\hline
a280 & 2 579 & 2 888 & 12.0\\
\hline
att48 & 10 628 & 11 196 & 5.34\\
\hline
att532 & 27 686 & 31 778 & 14.8\\
\hline
br17 & 39 & 77 & 97.4\\
\hline
brazil58 & 25 395 & 26 573 & 4.64\\
\hline
fl1577 & 22 204 & 27 736 & 24.9\\
\hline
fl3795 & 28 723 & 32 339 & 12.6\\
\hline
fnl4464 & 182 566 & 204 748 & 12.2\\
\hline
kroB100 & 22 141 & 24 687 & 11.5\\
\hline
korB150 & 26 130 & 31 117 & 19.1\\
\hline
kroB200 & 29 437 & 31 322 & 6.40\\
\hline
kroC100 & 20 749 & 23 184 & 11.7\\
\hline
kroD100 & 21 294 & 23 381 & 9.80\\
\hline
pla7397 & 23 260 728 & 26 840 880 & 15.4\\
\hline
pr2392 & 378 032 & 442 813 & 17.1\\
\hline
rl5915 & 565 040 & 665 876 & 17.8\\
\hline
rl5934 & 554 070 & 648 874 & 17.1\\
\hline
u1060 & 224 094 & 290 935 & 29.8\\
\hline
vm1084 & 239 297 & 270 554 & 13.1\\
\hline
vm1748 & 336 556 & 396 296 & 17.8\\
\hline
\end{tabular}
\end{center}

\end{document}
