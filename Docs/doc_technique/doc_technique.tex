\documentclass{article}

% Mise en page
\usepackage[scale=0.75]{geometry}
% Pied de page
\usepackage{fancyhdr}
\pagestyle{fancy}
\renewcommand{\headrulewidth}{0pt}
% Interligne après les paragraphes
\setlength{\parskip}{1.5ex}

% Langues
\usepackage[french]{babel}
\usepackage[utf8]{inputenc}
\usepackage[T1]{fontenc}

\usepackage{tabbing}

% Affichage pour les algos, numérotation en fonction des chapitres
\usepackage[ruled,vlined]{algorithm2e}
\SetAlgorithmName{Algorithme}{Liste des algorithme}

% Images
%\usepackage{graphicx}
%\usepackage{rotating}

\begin{document}

\newcommand{\HRule}{\rule{\linewidth}{0.5mm}}

\begin{titlepage}
\begin{center}

\includegraphics[width=0.5\textwidth]{polytech}~\\[1cm]

\textsc{\LARGE Programmation Stochastique}\\[1.5cm]

\textsc{\Large Projet d'implémentation de l'algorithme du Recuit Simulé}\\[0.5cm]

% Title
\HRule \\[0.4cm]
{ \huge \bfseries Traveling Salesman Problem \\[0.4cm] }
\HRule \\[1.5cm]

% Author and supervisor
\begin{minipage}{0.4\textwidth}
\begin{flushleft} \large
\emph{Auteur:}\\
Jeremie \textsc{Bosom}\\
Erwan \textsc{Chaussy}
\end{flushleft}
\end{minipage}
\begin{minipage}{0.4\textwidth}
\begin{flushright} \large
\emph{Professeur:} \\
M.Abdel \textsc{Lisser}
\end{flushright}
\end{minipage}

\vfill

% Bottom of the page
{\large \today}

\end{center}
\end{titlepage}


\rhead{Programmation Stochastique - Document technique}
\lfoot{\includegraphics[scale=0.3]{polytech.jpg}}
\rfoot{BOSOM - CHAUSSY}

\section*{Presentation du problème}

Le but de ce projet est de traiter le problème du voyager de commerce, Travelling Salesman Problem en anglais.
Il s'agit d'un problème de d'optimisation combinatoire qui a été formulé de manière générale en 1930, à l'aide d'un modèle mathématique clair.

Considérant un certain nombre de villes, le voyageur de commerce doit parcourir toutes les villes une et une seule fois.
On chercher le chemin le plus court qui part d'une ville, les parcours toutes et rejoins la ville de départ.

Cela semble simple et ça l'est, si l'on ne considère que des petites instances, c'est à dire une dizaine de villes.
En effet, si l'on a $n$ villes, il est possible de rejoindre $n-1$ et ainsi de suite.
On trouve donc un premier chemin en un temps proportionel à $O(n)$.
Mais pour être sur qu'il s'agit du plus court chemin, il faut examiner les autres chemins possible.
Si l'on cherche une solution exacte, on remarque que la résolution du problème se fait en un temps $O(n^{n})$
C'est ce que l'on considère être un problème NP-difficile.

Il semble évident que l'on ne peut pas énumérer toutes les solutions possible lorsque l'on traite 5000 villes par exemple.
Il faut donc imaginer des méthodes approchées qui permettent de trouver une solution plus proche de la solution optimale.
Dans ce projet, nous allons utiliser l'algorithme stochastique du recuit simulé que nous présenterons après le modèle mathématique du problème du voyageur de commerce.

\section*{Modèle mathématique}

Nous allons utiliser le jeu de donné suivant :\newline
\begin{itemize}
\item{$G = (V, E)$}{un graphe orienté complet de n sommets}
\item{$c_{ij}$}{le coût de l'arc $(v_{i}, v_{j})$}
\item{
$$
x_{ij} = \lbracs
\left . \begin{tabular}{l}
1 si et seulement si l'arc (i, j) est retenu dans le circuit, \\
0 sinon \\
\end{tabular}
\right
$$
}

\end{itemize}

Nous cherchons donc à minimiser le coût total du trajet. Cela se traduit par le programme linéaire suivant :

\begin{tabbing}
$$
\min\sum_{i=1}^{n}\sum_{j=1}^{n}c_{ij}x_{ij}//
s.t.\=\\
\+
\sum_{j=1}^{n}x_{ij} = 1, i = 1,\dots, n \\
\sum_{i=1}^{n}x_{ij} = 1, j = 1,\dots, n \\
\sum_{i|v_{i} \in S}\sum_{j|v_{j} \in S}x_{ij} \leq |S|-1
S \subset \{v_{1},\dots, v_{n}\} et S \neq \emptyset \\
x_{ij} \in \{0,1\}, 1 \leq i \leq n
$$
\end{tabbing}

La première contrainte décrit le fait qu'il n'y a qu'une et une seule arrête partant d'une ville.
La seconde contrainte décrit le fait qu'il n'y a qu'une et une seule arrête arrivant dans une ville.
La dernière contraite est la contrainte dite de sous-tour.
Elle empêche le fait que l'on passe plusieurs fois par une même ville.
C'est cette contrainte qui rend le problème du voyageur de commerce NP-difficile.
En effet, cette contrainte vérifie toutes les villes du graphe entres-elles, rendant le calcul extrèmement compliqué, réalisé en un temps en $O(n!)$.
Il devient évident qu'une recherche exhaustive ne peut être envisagée.

\section*{Recuit simulé}

L'algorithme que nous allons considéré est l'algorithme du recuit simulé.
Il s'agit d'une métaheuristique inspiré du processus utilisé en métallurgie.
%Le principe est d'explorer des voisinages.


\begin{algorithm}[H]
\SetAlgoLined
%\LinesNumbered
\Donnees{E le premier cycle généré
T la température du recuit}
\Res{S cycle hamiltonien }
\Deb{
	a complété
%	$S \leftarrow \emptyset$\;
%	\tcp{Ajout de la ville de départ}
%	Sélectionner $vDepart \in V$ aléatoirement\;
%	$v \leftarrow vDepart$\;
%	$V \leftarrow V \setminus \{vDepart\}$\;
%	\tcp{Parcours de toutes les villes pour l'ajout aléatoire}
%	\Tq{$V \neq \emptyset$}{
%		Sélectionner $v' \in V$ aléatoirement\;
%		$S \leftarrow S \cup \{v, v'\}$\;
%		$V \leftarrow V \setminus \{v\}$\;
%		$v \leftarrow v'$\;
%	}
%	$S \leftarrow S \cup \{v, vDepart\}$\;
%	\Retour S\;
}
\caption{Recuit simulé}
\end{algorithm}


\section*{Voisinage considéré}

A remplir

\section*{Réglage de la température du recuit simulé}

Le réglage de la température utilisée pour l'algorithme est quelque de compliqué puisque cela influe sur le resultat final.
En effet, une température trop haute acceptera trop de changement, rendant le début de l'algorithme inutile.
Au contraire, une température trop basse n'acceptera que les changements qui réduisent vraiment le coût total, au risque de bloquer dans un minimum local.
Il faut donc trouver la bonne température qui produira un taux d'acceptation suffisamment grand pour ne pas rendre le début de l'algorithme inutile.

En général, ce taux est fixé à au moins 80\% d'acceptation.
Nous exécutons donc l'algorithme du recuit simulé avec une température assez basse.
Au lieu de diminuer la température à la fin des itérations, nous vérifions si le taux est atteind ou non.
S'il ne l'est pas, nous doublons la température et recommençons.
Une fois le taux atteind, nous retournons la dernière température et lançons l'algorithme du recuit simulé.

\end{document}
